%%%%%%%%%%%%%%%%%
% This is an example CV created using altacv.cls (v1.1.4, 27 July 2018) written by
% LianTze Lim (liantze@gmail.com), based on the
% Cv created by BusinessInsider at http://www.businessinsider.my/a-sample-resume-for-marissa-mayer-2016-7/?r=US&IR=T
%
%% It may be distributed and/or modified under the
%% conditions of the LaTeX Project Public License, either version 1.3
%% of this license or (at your option) any later version.
%% The latest version of this license is in
%%    http://www.latex-project.org/lppl.txt
%% and version 1.3 or later is part of all distributions of LaTeX
%% version 2003/12/01 or later.
%%%%%%%%%%%%%%%%

%% If you want to use \orcid or the
%% academicons icons, add "academicons"
%% to the \documentclass options.
%% Then compile with XeLaTeX or LuaLaTeX.
% \documentclass[10pt,a4paper,academicons]{altacv}

%% Use the "normalphoto" option if you want a normal photo instead of cropped to a circle
% \documentclass[10pt,a4paper,normalphoto]{altacv}

\documentclass[9pt,a4paper]{altacv}

%% AltaCV uses the fontawesome and academicon fonts
%% and packages.
%% See texdoc.net/pkg/fontawecome and http://texdoc.net/pkg/academicons for full list of symbols.
%% When using the "academicons" option,
%% Compile with LuaLaTeX for best results. If you
%% want to use XeLaTeX, you may need to install
%% Academicons.ttf in your operating system's font %% folder.


% Change the page layout if you need to
\geometry{left=1cm,right=9cm,marginparwidth=6.8cm,marginparsep=1.2cm,top=1cm,bottom=1cm}

% Change the font if you want to.

% If using pdflatex:
\usepackage[utf8]{inputenc}
\usepackage[T1]{fontenc}
\usepackage[default]{lato}

% If using xelatex or lualatex:
% \setmainfont{Lato}
\usepackage{hyperref}

% Change the colours if you want to
\definecolor{PineGreen}{HTML}{01796F}
\definecolor{ForestGreen}{HTML}{014421}
\definecolor{SlateGrey}{HTML}{2E2E2E}
\definecolor{LightGrey}{HTML}{666666}

\colorlet{heading}{ForestGreen}
\colorlet{accent}{ForestGreen}
\colorlet{emphasis}{SlateGrey}
\colorlet{body}{LightGrey}

% Change the bullets for itemize and rating marker
% for \cvskill if you want to
\renewcommand{\itemmarker}{{\small\textbullet}}
\renewcommand{\ratingmarker}{\faCircle}

%% sample.bib contains your publications
\addbibresource{sample.bib}

\begin{document}
\name{\href{https://www.mahbubiftekhar.co.uk/}{Mahbub Iftekhar}}
\tagline{Final Year Computer Science Student}
% Cropped to square from https://en.wikipedia.org/wiki/Marissa_Mayer#/media/File:Marissa_Mayer_May_2014_(cropped).jpg, CC-BY 2.0
%\photo{2.5cm}{mmayer-wikipedia-cc-by-2_0}
\personalinfo{%
  % You can add your own with \printinfo{symbol}{detail}
                % Email

  \email{\href{mailto:mahbub2@iftekhar.co.uk}{mahbub2@iftekhar.co.uk}  }
  \phone{\href{tel:+447552695272}{+44 7552 69 5272}}
  \phone{\href{tel:+441315818775}{+44 131 581 8775}}
  \location{Edinburgh, United Kingdom}
  \homepage{\href{https://www.mahbubiftekhar.co.uk/}{https://www.mahbubiftekhar.co.uk/}}
  \linkedin{\href{https://www.linkedin.com/in/mahbub-iftekhar/}{https://www.linkedin.com/in/mahbub-iftekhar/}}
  \github{\href{https://github.com/mahbubiftekhar/}{https://github.com/mahbubiftekhar/}}

  % I'm just making this up though.
%   \orcid{orcid.org/0000-0000-0000-0000} % Obviously making this up too. If you want to use this field (and also other academicons symbols), add "academicons" option to \documentclass{altacv}
}

%% Make the header extend all the way to the right, if you want.
\begin{fullwidth}
\makecvheader
\end{fullwidth}

%% Depending on your tastes, you may want to make fonts of itemize environments slightly smaller
\AtBeginEnvironment{itemize}{\small}

%% Provide the file name containing the sidebar contents as an optional parameter to \cvsection.
%% You can always just use \marginpar{...} if you do
%% not need to align the top of the contents to any
%% \cvsection title in the "main" bar.

\cvsection[page1sidebar]{WORK}
\cvevent{Supervisor}{Shimla LTD (Shamoli Restaurant)}{August 2017 -- Ongoing}{Edinburgh, UK}
\begin{itemize}
\item Main tasks: Managing staff, liaising with customers and suppliers. Handling enquiries and administration regarding the day to day operations. Conducting repairs and maintenance of IT equipment on the premises. 
\divider
\end{itemize}

\cvevent{Computing Support Technician}{The University of Edinburgh}{May 2018 -- August 2018}{Edinburgh, UK}
\begin{itemize}
\item Main tasks: to work as part of a team, efficiently carrying out maintenance and repairs on University computer equipment; upgrading existing infrastructure; planning for, then installing new equipment to improve facilities.
\end{itemize}


\cvsection[page1sidebar]{PROJECTS}
%% https://play.google.com/store/apps/details?id=miftekhar.swipe2sort
%%\href{https://play.google.com/store/apps/details?id=miftekhar.swipe2sort}{Swipe 2 Sort - Final Year, Honours Project}.

\cvevent{\href{https://play.google.com/store/apps/details?id=miftekhar.swipe2sort}{Swipe 2 Sort - Final Year, Honours Project}}{Android Developer}{August 2018 -- May 2019}{Edinburgh, UK}
\begin{itemize}
\item Swipe 2 Sort is my final year project. A Kotlin Android app designed to make organising user's many photos simple and fun, it will be released on the Play Store towards the end of 2018.
\end{itemize}
\divider

%%\href{https://github.com/mahbubiftekhar/Children_with_Cancer_UK-JPMorganCodeForGood}{JP Morgan Code for Good - 2018}.

\cvevent{\href{https://github.com/mahbubiftekhar/Children_with_Cancer_UK-JPMorganCodeForGood}{JP Morgan Code for Good - 2018}
}{Database \& HTML Developer}{27th \& 28th of October 2018}  {Glasgow, UK}
\begin{itemize}
\item Code for Good is a 24 hour, competitive event which tasks a team to develop a solution for an NGO. My group developed a web application for Children with Cancer UK, which assists their patients to communicate with one another in a safe, user-friendly environment.
\end{itemize}
\divider

%%\href{https://github.com/mahbubiftekhar/RoboTour}{RoboTour - Autonomous Robot}.
\cvevent{\href{https://github.com/mahbubiftekhar/RoboTour}{RoboTour - Autonomous Robot}}{Team Leader \& Android Developer}{January 2018 -- May 2018}{Edinburgh, UK}
\begin{itemize}
\item The project involved working within a 7-strong group, designing and developing an autonomous robot from scratch, using an EV3 mind storm which was controlled by multiple Android Devices.
\end{itemize}
\divider

%%\href{https://github.com/mahbubiftekhar/quickSMS}{quickSMS - Kotlin Android App}.
\cvevent{\href{https://github.com/mahbubiftekhar/quickSMS}{quickSMS - Kotlin Android App}}{Android Developer \& Co-Creator}{December 2017 -- June 2018}{Edinburgh, UK}
\begin{itemize}
\item quickSMS is an Android App I co-developed with Alex Shand. quickSMS was written from scratch, using the emerging language, Kotlin. Studies were used to help design and improve the UI interface. The app has subsequently been published on the Google Play store for the public to enjoy.
\end{itemize}
\divider

%%\href{https://github.com/mahbubiftekhar/Songle}{Songle - Kotlin Android App}.
\cvevent{\href{https://github.com/mahbubiftekhar/Songle}{Songle - Kotlin Android App}}{Android Developer}{September 2017 -- December 2017}{Edinburgh, UK}
\begin{itemize}
\item The very first Android app I developed, Songle is a location-based game, where the user walks around collecting words in the University of Edinburgh central campus , obtaining the words by going close to markers.
\end{itemize}
\end{document}